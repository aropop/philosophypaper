\documentclass[a4paper]{article}

\usepackage[english]{babel}
\usepackage{hyperref}
\usepackage[utf8]{inputenc}
\usepackage{vub}
\usepackage{todonotes}
\usepackage{color}
\usepackage{tabularx}
\usepackage{listings}
\usepackage{graphicx}
\usepackage{hyperref}
\usepackage{a4wide}
\usepackage[nounderscore]{syntax}

\author{Arno~De~Witte (0500504)}
\title{Are we real or imaginary?}
\subtitle{Philosophy of science}



\lstset{
language=php,
numbers=left,
frame=single,
columns=flexible,
showstringspaces=true,
tabsize=3,
keepspaces=true,
basicstyle=\ttfamily,
breaklines=true,
  basicstyle=\small\ttfamily,
  keywordstyle=\bf\ttfamily\color[rgb]{0,.3,.7},
  commentstyle=\color[rgb]{0.133,0.545,0.133},
  stringstyle={\color[rgb]{0.75,0.49,0.07}}
}

% Nodig voor nieuwe paragrafen "correct" te laten beginnen
\setlength{\parskip}{\medskipamount}
\setlength{\parindent}{0ex}

\begin{document}
\makeassignment
\section{Introduction}
The idea behind brain in a vat has been around for a long time. The brain in a vat thought-experiment describes a brain that is kept in vat and whose neurons are being controlled by a very powerful computer. Thus all your senses, which you assume are real, are just a simulation created by a computer program. The idea was brought to the mainstream with the popular movie \emph{The Matrix} by the Wachowski brothers in 1999. Back then this seemed a far fetched idea, but with the current advancements in artificial intelligence and the continuous increase in computing power, is this still a thought-experiment or can it become reality?\\

This paper will elaborate the brain in a vat thought experiment, the advancements in artificial intelligence, whether we can (and should) create a brain in a vat and the possibility that we ourselves are a brain in vat.

\section{Background}
\subsection{Idea}
The brain in a fat thought-experiment is a type of the simulated reality. It was introduced by Harman \cite{Harman15} in 1973 as an updated version of Réné Descartes' evil demon (or evil genius). This idea describes a demon which alters the senses of a person. Thus creating the illusion of the world, his thoughts and his own body. The idea of a deceptive creator or god is called Deus Desceptor. This caused for claims of blasphemy against Descartes.\\

The idea that the things we sense cannot be trusted is called a sceptical hypotheses. Another example is the five minute world, which states that the world is only created recently with records and indications that makes it look like it is way older.\\
This sceptical hypotheses is part of a school in philosophy which is called the sceptical philosophy. Which core idea is that it is not possible to have a adequate judgement of what is true.

\subsection{Advancements In Artificial Intelligence}
Artificial intelligence is defined as intelligence inherited by a machine \cite{AIbasic}. Artificial intelligence has many branches, for example game theory, where programs are written to beat certain games such as chess. An important branch of artificial intelligence is machine learning. In machine learning algorithms are defined that are able to learn a solution to a problem given a set of test data. For example learning how to balance a stick, by rewarding it when it succeeds and punishing it when it fails (this algorithm is called reinforcement learning). One of the key differences in machine learning is that the algorithms don't define how to solve a problem, but how to learn a solution for the problem.\\

In the branch of machine learning many of the algorithms are based on how we as humans are built. For example the neural network algorithm is based on how our brains work \cite{Nnetwork}. It has a set of inputs or senses, then a set of intermediate nodes (neurons) which send signals to each other and a set of outputs. To construct such a network several different methods exist, from using training data and test data to achieve highest coverage, to reinforcement training techniques (punish and reward). However there is also genetic algorithms. These will create strains of algorithms with a certain amount of nodes and node placements and try to modify these slightly. Thus these networks will evolve over time to become better suited to solving a problem. A nice example of this is MarI/O \cite{Mario}, which is a evolutionary neural network trained to play the classic Mario game.\\

Many fascinating examples can be found using these kinds of techniques and machine learning is finding its way into many applications. It is said that 50\% of stock trade is accounted by smart computers \cite{Stocks}, speech and face recognition, business intelligence, \dots\\
Some might argue that computers lack creativity, the ability to create art and music. However recent trends in machine learning show that even computers are capable of creating pieces of art and music \cite{music}.\\
Too these problems we cannot find an adequate solution ourselves. We just let a computer learn a solution for us. The model or network that this computer created, is something we can try to analyse and try to find an explanation for. But with increasing complexity of these models or networks, this has become harder and almost impossible.\\
What we can learn from this is that maybe someone created us as a computer program to solve a problem he faced. We became smarter, we learned and maybe some day we might find a solution for his problem. A solution which he may not even understand himself, but a solution that will solve his problem.
 
 
\section{Criticism}
The idea of us being a brain in vat, or that we would be able to create a brain in vat has gotten a lot of criticism. From a philosophical standpoint and from a technical standpoint.

\subsection{Philosophical reasons}
Many philosophical research has gone into this topic. Hilary Putnam has argued that the brain in a vat scenario cannot be true \cite{Putnam}. His argument states that if we (and we are not in vat) refer to something we know the real concept. If I say apple, I know a real apple, whilst if a brain in a vat says apple it can only reference to something. He calls this a casual constraint, for which he uses the example of an ant which (accidentally) draws the face of Winston Churchill. The ant does not know who or what Winston Churchill, thus it can impossibly reference to him.\\
His argument goes as follows: Assume that we are brains in vat, then brains and vat do not reference to the ``real" brains and ``real" vat. The sentence ``we are brains in a vat" is thus false. So we can conclude, if we are brains in a vat, the sentence we are brains in a vat cannot be true.\\

\subsection{Complexity}
The technical discussion of this is simpler than the philosophical discussion. At this moment in time we do not have enough understanding of the brain to be able to sustain a brain in a vat on a nutrient level. So creating a physical brain in vat is something that is not possible at this time and is questioned to be possible in the future. However this argument might be irrelevant, why would a super human create physical brains in vat? It would require a vast amount of resources: nutrients, fibbers, water, etc.\\
However simulating one would be simpler. If we want to build a bridge and check its balance, we create a simulation because building a model is time consuming and error prone. The same goes for a brain in vat. Creating a computer program is simple and does not require any of the maintenance physical brains require.\\
There are however some problems with simulation as well. The most obvious one would be raw computing power. Currently the most powerful computers are used to process large amounts of data. However the amount of data we process as a human is bigger than that, most neural networks work on only a fraction of the data we process. The reaction time far superior in some domains, think about facial recognition and speaking languages.\\
It is clear that creating an artificial intelligence, that has the same capabilities as us is difficult. Certainly if we want to solve a specific problem, it would be wise to constrain the amount of inputs and outputs of our smart agents. This implies that if we were to be brains in vat, the simulation we are living in is far less complicated then the world outside of the vat. Is it thus possible for us to either be able to prove that we are in fact a brain in vat and to comprehend the world outside of the vat?\\

\subsection{Is it possible and ethical?}
There are however philosophers who claim that this brain in vat theory is possible \cite{IsPossible}. However state that it is unlikely to happen. As there are some ethical and technical problems that arise. If we create a brain in vat that has become conscious through evolving over time, can we just shut it off? Is creating artificial life and managing it, the same as playing god? We morally object against killing animals for testing because these animals suffer pain and have emotions. But when we create an artificial intelligence that has emotions and can also feel pain (which could be an reinforcing characteristic of the algorithm), does shutting it down imply the same morals?\\
As far as the technical problem, what if the artificial intelligence finds out it is an artificial intelligence? Should we modify the algorithm to restrict certain conclusion and what are the consequences of such restrictions? These are questions that should be thought about and are similar to ethical problems that arise with certain medical advancements. However the answers on these questions remain vastly unknown. 

\section{Personal Opinion}
I think that the brain in a vat experiment has two main parts that we should consider: the fact that we might be a brain in a vat ourselves on the one hand and the possibilty of creating a brain in vat ourselves on the other hand.\\

To me the first part is something that we might one day discover, however I think that the chances are not on our side. If we were a brain in a vat, we would most likely be either totally different from our creators or be way dumber. Either way if we would find evidence that we were in fact a brain in vat (I think most likely in somewhere in the quantum realm), would we be able to see it? I think most likely not. If we were in fact created to solve a problem, we would most likely be programmed in a way that would reinforce steps in solving the problem and would punish steps backwards.\\
The personal aspect of this theory also raises some more questions for me. If I were to be brain in vat, are the people I'm seeing also brains in a vat, or are these people just simulations that guide me in the direction of the solution? If I were to be a brain in a vat and considering I'm here to solve a problem, what if I cannot find the answer, will my existence be switched off? The ethical questions that apply for us in creating the brain in a vat, do these also apply to our creators? For me the fact that we are a brain in a vat is very similar to the existence of an all knowing god. The god might punish you for doing bad and will reward you if you are doing good. It is almost impossible to prove that he exists, but he can explain everything there is to know.

Existentinal issues, why are we here, what is the purpose, if im just prgram why should i do anything....
Comparison to god

\begin{thebibliography}{9}

\bibitem{Harman15}
  Harman, G. (2015). \emph{Thought}. Princeton University 

\bibitem{Putnam}
  Putnam, H. (1981). \emph{Brains in a Vat}.
  
\bibitem{Mario}
  SethBling. (2015) \emph{MarI/O}. \url{https://www.youtube.com/watch?v=qv6UVOQ0F44} retrieved on 25th may 2016.
  
\bibitem{AIbasic}
Russell, S., Norvig, P., \& Intelligence, A. (1995). \emph{A modern approach. Artificial Intelligence}. Prentice-Hall, Egnlewood Cliffs, 25, 27.

\bibitem{Nnetwork}
Russell, I. (1996) \emph{Definition of a Neural Network}. \url{http://uhaweb.hartford.edu/compsci/neural-networks-definition.html} retrieved on 26th may 2016.

\bibitem{Stocks}
 Iati, R. (2009) \emph{The Real Story of Trading Software Espionage}. AdvancedTrading.com
 
\bibitem{IsPossible}
 Heylighen, F. (2012). \emph{Brain in a vat cannot break out}. Journal of Consciousness Studies, 19(1-2), 126-142.
 
\bibitem{music}
Futurism. (2016). \emph{Can an Artificial Intelligence Create Art?}. \url{http://futurism.com/can-an-artificial-intelligence-create-art-googles-making-an-ai-to-find-out/} retrieved on 26th of may 2016.


\end{thebibliography}
\end{document}
