\documentclass[a4paper]{article}

\usepackage[english]{babel}
\usepackage{hyperref}
\usepackage[utf8]{inputenc}
\usepackage{vub}
\usepackage{todonotes}
\usepackage{color}
\usepackage{tabularx}
\usepackage{listings}
\usepackage{graphicx}
\usepackage{hyperref}
\usepackage{a4wide}
\usepackage[nounderscore]{syntax}

\author{Arno~De~Witte (0500504)}
\title{Are we real or imaginary?}
\subtitle{Philosophy of science}



\lstset{
language=php,
numbers=left,
frame=single,
columns=flexible,
showstringspaces=true,
tabsize=3,
keepspaces=true,
basicstyle=\ttfamily,
breaklines=true,
  basicstyle=\small\ttfamily,
  keywordstyle=\bf\ttfamily\color[rgb]{0,.3,.7},
  commentstyle=\color[rgb]{0.133,0.545,0.133},
  stringstyle={\color[rgb]{0.75,0.49,0.07}}
}

% Nodig voor nieuwe paragrafen "correct" te laten beginnen
\setlength{\parskip}{\medskipamount}
\setlength{\parindent}{0ex}

\begin{document}
\makeassignment
\section{Introduction}
The idea behind brain in a vat has been around for a long time. The brain in a vat thought-experiment describes a brain that is kept in vat and whose neurons are being controlled by a very powerful computer. Thus all your senses, which you assume are real, are just a simulation created by a computer program. The idea was brought to the mainstream with the popular movie \emph{The Matrix} by the Wachowski brothers in 1999. Then this seemed a far fetched idea, but with the advancements in artificial intelligence and the continuous increase in computing power, is this still a thought-experiment or can it become reality?\\

This paper will elaborate the brain in a vat thought experiment, the advancements in artificial intelligence, whether we can (and should) create a brain in a vat and the possibility that we ourselves are a brain in vat.

\section{Background}
\subsection{Idea}
The brain in a fat thought-experiment is a type of the simulated reality. It was introduced by Harman \cite{Harman15} in 1973 as an updated version of Réné Descartes' evil demon (or evil genius). This idea describes a demon which alters the senses of a person. Thus creating the illusion of the world, his thoughts and his own body. The idea of a deceptive creator or god is called Deus Desceptor. This caused for claims of blasphemy against Descartes.\\

The idea that the things we sense cannot be trusted is called a sceptical hypotheses. Another example is the five minute world, which states that the world is only created recently with records and indications that makes it look like it is way older.\\
This sceptical hypotheses is part of a school in philosophy which is called the sceptical philosophy. Which core idea is that it is not possible to have a adequate judgement of what is true.

\subsection{Advancements In Artificial Intelligence}
Artificial intelligence is defined as intelligence inherited by a machine \cite{AIbasic}. Artificial intelligence has many branches, for example game theory, where programs are written to beat certain games such as chess. An important branch of artificial intelligence is machine learning. In machine learning algorithms are defined that are able to learn a solution to a problem given a set of test data. For example learning how to balance a stick, by rewarding it when it succeeds and punishing it when it fails (this algorithm is called reinforcement learning). One of the key differences in machine learning is that the algorithms don't define how to solve a problem, but how to learn a solution for the problem.\\

In the branch of machine learning many of the algorithms are based on how we as humans are built. For example the neural network algorithm is based on how our brains work. It has a set of inputs or senses, then a set of intermediate nodes which send signals to each other and a set of outputs.


What is AI and machine learning, used to solve problems we don't have a trivial answer for
Neural networks -> structured like brain
Evolutionary machine learning
mar IO example



 
 
\section{Criticism}
The idea of us being a brain in vat, or that we would be able to create a brain in vat has gotten a lot of criticism. From a philosophical standpoint and from a technical standpoint.
\subsection{Philosophical reasons}
Many philosophical research has gone into this topic. Hilary Putnam has argued that the brain in a vat scenario cannot be true. His argument states that if we (and we are not in vat) refer to something we know the real concept. If I say apple, I know a real apple, whilst if a brain in a vat says apple it can only reference to something. Therefor it cannot have any real knowledge only references which are fed by the machine controlling it. It is thus impossible that the brain in a vat knows about a brain in vat.\\

\subsection{Complexity}
Huge amount of computing power
Simplification of existing world -> would we become smarter if necessary

\subsection{Is it possible?}
There are however philosophers who claim that this brain in vat theory is possible. However state that it is unlikely to happen. As there are some ethical, technical and 

\section{Personal Opinion}
Possible -> certainly, why?

Existentinal issues, why are we here, what is the purpose, if im just prgram why should i do anything....
Comparison to god

\begin{thebibliography}{9}

\bibitem{Harman15}
  Harman, G. (2015). \emph{Thought}. Princeton University 

\bibitem{Putnam}
  Putnam, H. (1981). \emph{Brains in a Vat}.
  
\bibitem{Mario}
  SethBling. (2015) \emph{MarI/O}. \url{https://www.youtube.com/watch?v=qv6UVOQ0F44} retrieved on 25th may 2016.
  
\bibitem{AIbasic}
Russell, S., Norvig, P., \& Intelligence, A. (1995). \emph{A modern approach. Artificial Intelligence}. Prentice-Hall, Egnlewood Cliffs, 25, 27.

\bibitem{Nnetwork}
Russell, I. (1996) \emph{Definition of a Neural Network}. \url{http://uhaweb.hartford.edu/compsci/neural-networks-definition.html} retrieved on 26th may 2016.

\end{thebibliography}
\end{document}
